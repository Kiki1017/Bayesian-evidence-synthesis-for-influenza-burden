\documentclass{article}[12pt]
\usepackage{amsmath,amssymb,authblk,bm,bm,caption,graphicx,hyperref,lscape,mathrsfs,setspace,subcaption,tabularx,tkz-graph,upgreek,url}
\usetikzlibrary{shapes.geometric}
\usepackage[letterpaper, margin=2.5cm]{geometry}
\usepackage[backend=biber,style=numeric,isbn=false,sorting=none,giveninits=true,terseinits=true]{biblatex}
\DeclareNameAlias{default}{last-first}
%\doublespacing
%\graphicspath{{ORgraphics/}}
\newcommand{\la}{\leftarrow}
\newcommand{\ra}{\rightarrow}
%\usepackage{draftwatermark}
\newcommand{\footremember}[2]{%
	\footnote{#2}
	\newcounter{#1}
	\setcounter{#1}{\value{footnote}}
}
\newcommand{\footrecall}[1]{%
	\footnotemark[\value{#1}]%
}

\newcommand{\absdiv}[1]{%
	\par\addvspace{.5\baselineskip}% adjust to suit
	\noindent\textbf{#1}\quad\ignorespaces
}

\renewcommand*{\revsdnamepunct}{}

\makeatletter
\renewbibmacro*{name:last-first}[4]{%
	\ifuseprefix
	{\usebibmacro{name:delim}{#3#1}%
		\usebibmacro{name:hook}{#3#1}%
		\ifblank{#3}{}{%
			\ifcapital
			{\mkbibnameprefix{\MakeCapital{#3}}\isdot}
			{\mkbibnameprefix{#3}\isdot}%
			\ifpunctmark{'}{}{\bibnamedelimc}}%
		\mkbibnamelast{#1}\isdot
		\ifblank{#4}{}{\bibnamedelimd\mkbibnameaffix{#4}\isdot}%
		%      \ifblank{#2}{}{\addcomma\bibnamedelimd\mkbibnamefirst{#2}\isdot}}% DELETED
		\ifblank{#2}{}{\bibnamedelimd\mkbibnamefirst{#2}\isdot}}% NEW
	{\usebibmacro{name:delim}{#1}%
		\usebibmacro{name:hook}{#1}%
		\mkbibnamelast{#1}\isdot
		\ifblank{#4}{}{\bibnamedelimd\mkbibnameaffix{#4}\isdot}%
		%      \ifblank{#2#3}{}{\addcomma}% DELETED
		\ifblank{#2}{}{\bibnamedelimd\mkbibnamefirst{#2}\isdot}%
		\ifblank{#3}{}{\bibnamedelimd\mkbibnameprefix{#3}\isdot}}}
\makeatother


%\SetWatermarkText{Confidential}
%\SetWatermarkScale{5}
% Sets the default location of pictures\fvset{fontsize=\normalsize} % The font size of all verbatim text can be changed here
%
\DeclareMathOperator{\Ex}{\mathbb{E}}% expected value
\DeclareMathOperator{\Poi}{\operatorname{Poi}}
%
\addbibresource{biblio.bib}
\title{} 
%
\author[1,2,*]{Ivo M Foppa}
%
\author[2]{\ldots}
\affil[1]{Battelle Memorial Institute, Atlanta, Georgia, USA}
\affil[2]{Influenza Division, Centers for Disease Control and Prevention, 1600 Clifton Road NE, Atlanta, 30333 Georgia, USA}
\affil[*]{Corresponding Author, Influenza Division, Centers for Disease Control and Prevention, 1600 Clifton Road NE, MS A-20, Atlanta, 30333 Georgia, USA, \nolinkurl{vor1@cdc.gov}}
\date{}
%
%----------------------------------------------------------------------------------------
\begin{document}
	{\let\newpage\relax\maketitle}	
	\maketitle%
	%
	\subsection*{Data} 
	%
\begin{enumerate}
	\item $N$: Total FluSurv-NET (FSN) population (given stratum, e.g. age group etc.)
	\item $M$: Total US population 
	\item $n_{H}$: Number of observed influenza hospitalizations with non-lethal outcome
	\item $n^\ast_H$: Number of total (observed and unobserved) influenza hospitalizations with non-lethal outcome
	\item $\lambda_H$: Rate of non-lethal flu hospitalizations per population
	\item $n_D$: Observed influenza deaths
	\item $n^\ast_D$: Total (observed and unobserved) influenza deaths
	\item $p_{k}$: Probability influenza-associated outcomes ($k=0$: non-lethal, $k=1$: lethal) that are correctly attributed to influenza
	\item $t_{k,j}$: Numbers tested by outcome and test type (1: PCR, 2: Rapid, 3: Other, 4: No test)
	\item $\rho_{k}$: Prior dist. for test sensitivities (PCR, rapid; mean, SD) by outcome
\end{enumerate}
	%

	\begin{align}
	n^\ast_H \sim & \frac{\left(\lambda_H N\right)^{n^\ast_H} e^{-\lambda_H N}}{\left(\lambda_H N\right)!}\\
	n_H \sim & \binom{n^\ast_H}{n_H} p_0^{n_H} \left(1 - p_0\right)^{n_H - n_H}
	\end{align}
	% ## The # of flu hosp in FSN with not fatal outcome; unobserved

%	\printbibliography
\end{document} 
rfludeathish <- rfludeath*(1-posh)

fludeath ~ dpois(rfludeathish*FSNpop) %## The total # of flu deaths in FSN; unobserved

FSNfludeath ~ dbin(pt[2],fludeath) %## Observed deaths among hospitalized influenza patents

rfluhosp ~ dunif(0,10)
rfludeath ~ dunif(0,10)

%for (k in 1:2){
%	## pt[k] is the prob. of influenza to be detected, as a function of testing prob. und sensitivity (by outcome)
%	## The three sensitivities, test probs and flu risks correspond to the 3 diff. tests;
%	## the 4th flu risk and 'test prob' represents those not tested

%	pt[k] <- ptest[k,1]*sens[k,1] + ptest[k,2]*sens[k,2] + ptest[k,3]*sens[k,3]

%	nttype[k,] ~ dmulti(ptest[k,1:4],ntot[k])   ### Observed testing freqs

%	ptest10[k] ~ dunif(0,1)
ptest20[k] ~ dunif(0,1)
ptest30[k] ~ dunif(0,1)
ptest40[k] ~ dunif(0,1)
ptot[k] <- ptest10[k] + ptest20[k] + ptest30[k] + ptest40[k]

ptest[k,1] <- ptest10[k]/ptot[k]
ptest[k,2] <- ptest20[k]/ptot[k]
ptest[k,3] <- ptest30[k]/ptot[k]
ptest[k,4] <- ptest40[k]/ptot[k]

pflu[k] ~ dunif(0,1)

for (m in 1:3) {
	flupos[k,m] ~ dbin(pflu[k],nttype[k,m]) %## actual flu pos/test type, unobserved
	testpos[k,m] ~ dbin(sens[k,m],flupos[k,m])% ## observed flu pos/test type
}

%	## generating a sensitvity from the given distribution
sens1[k] ~ dnorm(pcrsens[1],1/(pcrsens[2]*pcrsens[2]))
logsens2[k] ~ dnorm(lrapidsens[1],1/(lrapidsens[2]*lrapidsens[2]))I(,0)
sens2[k] <- exp(logsens2[k])
sens3[k] ~ dunif(0,1)

sens[k,1] <- sens1[k]
sens[k,2] <- sens2[k]
sens[k,3] <- sens3[k]
}

USfluhosp ~ dpois((rfluhosp + rfludeathish)*Npop) %## The total # of US flu hosp, non-fatal + fatal

USfludeath ~ dpois(rfludeath*Npop) ## The total %# of US flu deaths
}
