\documentclass{article}[12pt]
\usepackage{amsmath,amssymb,authblk,bm,bm,caption,graphicx,hyperref,lscape,mathrsfs,setspace,subcaption,tabularx,
	tkz-graph,upgreek,url}
\usetikzlibrary{shapes.geometric}
\usepackage[letterpaper, margin=2.5cm]{geometry}
\usepackage[backend=biber,style=numeric,isbn=false,sorting=none,giveninits=true,terseinits=true]{biblatex}
\DeclareNameAlias{default}{last-first}
%\doublespacing
%\graphicspath{{ORgraphics/}}
\newcommand{\la}{\leftarrow}
\newcommand{\ra}{\rightarrow}
%\usepackage{draftwatermark}
\newcommand{\footremember}[2]{%
	\footnote{#2}
	\newcounter{#1}
	\setcounter{#1}{\value{footnote}}
}
\newcommand{\footrecall}[1]{%
	\footnotemark[\value{#1}]%
}

\newcommand{\absdiv}[1]{%
	\par\addvspace{.5\baselineskip}% adjust to suit
	\noindent\textbf{#1}\quad\ignorespaces
}

\renewcommand*{\revsdnamepunct}{}

\makeatletter
\renewbibmacro*{name:last-first}[4]{%
	\ifuseprefix
	{\usebibmacro{name:delim}{#3#1}%
		\usebibmacro{name:hook}{#3#1}%
		\ifblank{#3}{}{%
			\ifcapital
			{\mkbibnameprefix{\MakeCapital{#3}}\isdot}
			{\mkbibnameprefix{#3}\isdot}%
			\ifpunctmark{'}{}{\bibnamedelimc}}%
		\mkbibnamelast{#1}\isdot
		\ifblank{#4}{}{\bibnamedelimd\mkbibnameaffix{#4}\isdot}%
		%      \ifblank{#2}{}{\addcomma\bibnamedelimd\mkbibnamefirst{#2}\isdot}}% DELETED
		\ifblank{#2}{}{\bibnamedelimd\mkbibnamefirst{#2}\isdot}}% NEW
	{\usebibmacro{name:delim}{#1}%
		\usebibmacro{name:hook}{#1}%
		\mkbibnamelast{#1}\isdot
		\ifblank{#4}{}{\bibnamedelimd\mkbibnameaffix{#4}\isdot}%
		%      \ifblank{#2#3}{}{\addcomma}% DELETED
		\ifblank{#2}{}{\bibnamedelimd\mkbibnamefirst{#2}\isdot}%
		\ifblank{#3}{}{\bibnamedelimd\mkbibnameprefix{#3}\isdot}}}
\makeatother


%\SetWatermarkText{Confidential}
%\SetWatermarkScale{5}
% Sets the default location of pictures\fvset{fontsize=\normalsize} % The font size of all verbatim text can be changed here
%
\DeclareMathOperator{\Ex}{\mathbb{E}}% expected value
\DeclareMathOperator{\Poi}{\operatorname{Poi}}
%
\addbibresource{biblio.bib}
\title{Bayesian Evidence Synthesis for Influenza Burden Estimation from Hospitalization Surveillance data} 
%
\author[1,2,*]{Ivo M Foppa}
%
\date{May 10, 2019}
%
%----------------------------------------------------------------------------------------
\begin{document}
	{\let\newpage\relax\maketitle}	
	\maketitle%
	%
	\subsection*{Notation (observed data/assumtions in \textbf{bold})} 
	%
\begin{enumerate}
	\item\textbf{ $N$: Total FluSurv-NET (FSN) population (given stratum, e.g. age group etc.)}
\item $n_H$: Number of total (observed and unobserved) influenza hospitalizations with non-lethal outcome
\item \textbf{$n^\ast_{H}$: Number of observed influenza hospitalizations with non-lethal outcome}
\item $\lambda_H$: Rate of non-lethal flu hospitalizations per population
\item $n_D$: Total (observed and unobserved) influenza deaths
\item\textbf{ $n^\ast_D$: Observed influenza deaths}
\item $\lambda_D$: Rate of lethal flu hospitalizations per population
\item $p_{k}$: Probability influenza-associated outcomes ($k=0$: non-lethal, $k=1$: lethal) that are correctly attributed to influenza
\item\textbf{ $T_{k,j}$: Numbers tested by outcome and test type (1: PCR, 2: Rapid, 3: Other, 4: No test)}
\item \textbf{$\rho_{k}$: Prior dist. for test sensitivities (PCR, rapid; mean, SD) by outcome}
\item $\pi$: Proportion of deaths outside hospital
\item \textbf{$D$: Total NCHS deaths (per syndromic cause and stratum)}
\item \textbf{$D_H$: NCHS deaths inside hospital (per syndromic cause and stratum)}
\end{enumerate}
	%
	\clearpage
	\subsection*{Model} 
%
\begin{figure}[h]
	\centering
	\resizebox{.5\textwidth}{!}{
		\scalebox{3}{
	\begin {tikzpicture}[>= stealth,scale=2.2]
	
	\node (N)  at (0.5,1.5) {$N$};
	\node (lamH)  at (0,1.5) {$\lambda_H$};
    \node (lamD) at (1,1.5) {$\lambda_D$};
	\node (nH) at (0,1) {$n_H$};
	\node (nHast)[rectangle,draw] at (0,.5) {$n^\ast_H$};%observed
	\node (nD) at (1,1) {$n_D$};
	\node (nDast)[rectangle,draw] at (1,.5) {$n^\ast_D$};%observed
	
	\node(A2a) at (0,.75) {} ;
	\node(A2b) at (1,.75) {} ;
	
	\node(pi) at (1.5,.75) {$\pi$} ; %proportion deaths ish
	\node(D)[rectangle,draw] at (2,1) {$D$} ; %all deaths NCHS
	\node(DH)[rectangle,draw] at (2,.5) {$D_H$} ; %all deaths NCHS
	\node(A3) at (2,.75) {} ;


	\node (pt) at (.5,.75) {$p_k$}; %prob. detection
	\node (phi) at (1.,0) {$\phi_{k,t}$}; % test dist 
	\node (T)[rectangle,draw] at (.5,0) {$T_{k,t}$}; % observed, # number tests
	\node (sens) at (0,0) {$\sigma_{k,t}$}; % test sens
	\node (rho)[ellipse,inner sep=0.5pt,draw] at (-.5,.25) {$\rho_{k,t}$}; % prior for sens 
	
	\node (pf) at (0,-.25) {$f_k$}; % true flupos
	\node (nF)[rectangle,draw] at (.5,-.5) {$F_{k,t}$}; %observed flu positives


	\draw[->](N) edge (nH);
	\draw[->](N) edge (nD);

	\draw[->](lamH) edge (nH);
	\draw[->](lamD) edge (nD);
	\draw[->](nH) edge (nHast);
\draw[->](nD) edge (nDast);
\draw[->] (pt) edge (A2a);
\draw[->] (pt) edge (A2b);

\draw[->] (pi) edge (A2b);
\draw[->] (pi) edge (A3);
\draw[->] (D) edge (DH);

\draw[->] (rho) edge (sens) ;
\draw[->] (sens) edge (pt) ;
\draw[->] (phi) edge (pt) ;

\draw[->] (T) edge (nF) ;

	\node(A4) at (.5,-.25) {} ;
\draw[->] (pf) edge (A4) ;
\draw[->] (phi) edge (T) ;
\draw[->] (sens) edge (A4) ;


	\end {tikzpicture}
}
%	\draw[->](lamH) edge [out=300,in=180] (nH);

%	\draw[->,line width=.8pt,color=red](nH) edge (nHast);
}
	\caption{Model structure. Rectangles represent observed data, circles are latent variables (unobserved data)and the ellipse represents an informative prior. Vague priors were omitted.}
	\label{fig2}
\end{figure}
%
\subcaption*{Modeling test sensitivity $\sigma_{k,t}$}

%	\printbibliography
\end{document} 